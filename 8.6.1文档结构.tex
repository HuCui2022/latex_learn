%导言区
\documentclass{ctexart} %ctexbook ctexrep,ctexart
%\usepackage{ctex}
%设置标题格式。
\ctexset{
	section = {
		format+ = \zihao{-4} \heiti \raggedright,
		name = {,、},
		number = \chinese{section},
		beforeskip = 1.0ex plus 0.2ex minus .2ex,
		afterskip = 1.0ex plus 0.2ex minus .2ex,
		aftername = \hspace{0pt}
	},	
	subsection = {
		format+ = \zihao{5} \heiti \raggedright,
		%name = {\thesubsection、}
		name = {,、},
		number = \arabic{subsection},
		beforeskip = 1.0ex plus 0.2ex minus .2ex,
		afterskip = 1.0ex plus 0.2ex minus .2ex,
		aftername = \hspace{0pt}
	}
}




%正文区

\begin{document}

	\section{引言}	
	故事:在现实认知观的基础上,对其描写成非常态性现象。是文学体裁的一种,侧重于事件发展过程的描述。强调情节的生动性和连贯性,较适于口头讲述。已经发生事。或者想象故事。
	
	故事一般都和原始人类的生产生活有密切关系,他们迫切地希望认识自然,于是便以自身为依据,想象天地万物都像人一样,有着生命和意志。			
	故事:在现实认知观的基础上,对其描写成非常态性现象。是文学体裁的一种,侧重于事件发展过程的描述。强调情节的生动性和连贯性,较适于口头讲述。已经发生事。或者想象故事。\par 故事一般都和原始人类的生产生活有密切关系,他们迫切地希望认识自然,于是便以自身为依据,想象天地万物都像人一样,有着生命和意志。
	
	\section{实验方法}	
	\section{实验结果}	
	\subsection{数据}
	\subsection{图表}
	\subsubsection{实验条件}
	\subsubsection{实验过程}
	\subsection{结果分析}
	\section{结论}	
	\section{致谢}	
	
\end{document}