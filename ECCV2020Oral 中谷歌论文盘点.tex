%导言区
\documentclass{article}

\usepackage{ctex}
\title{ECCV 2020 Oral 中谷歌论文盘点,点云与3D方向工作居多}
\date{\today}

\usepackage[a4paper,left=10mm,right=10mm,top=15mm,bottom=15mm]{geometry} 
\usepackage{listings}
\setlength{\parindent}{0pt}% 去掉首行缩进。
\usepackage[center]{titlesec}%设置标题格式。
%\titleformat{\section}{\centering}
\titleformat{\section}{\centering\Huge\bfseries}{第\,\thechapter\,章}{1em}{}
\begin{document}
	\maketitle
	\begin{abstract}
		本文汇总其入选 ECCV 2020 Oral 的作品,总计 10 篇。这些论文分布在神经渲染、点云、3D人体重建、3D场景重建、人类行为预测、光流、自监督、6D位姿估计等领域,特别值得一提的是「NeRF: Representing Scenes as Neural Radiance Fields for View Synthesis 」获得最佳论文荣誉提名!其展示效果让人惊艳。
	\end{abstract}
	\section*{最佳论文荣誉提名}
		
	   \begin{itemize}%enumerate/description
	   	\item
	   	[1].NeRF: Representing Scenes as Neural Radiance Fields for View Synthesis\\
	   	作者 | Ben Mildenhall、Pratul P. Srinivasan、Matthew Tancik、Jonathan T. Barron、Ravi\\ Ramamoorthi、Ren Ng\\\
	   	单位 | UC 伯克利;谷歌;加州大学圣地亚哥分校\\
	   	论文 | https://arxiv.org/abs/2003.08934\\
	   	主页 | https://www.matthewtancik.com/nerf\\
	   	备注 | ECCV 2020 Oral\\
	   	解读 | https://zhuanlan.zhihu.com/p/187541908\\
	   	
	   	发明了一种称之为神经辐射场的方法用于场景3D新视野的合成,输入不同视角下拍的图片,合成未出现的视角的图像。\\
	   	\item 
	   	[2].Quaternion Equivariant Capsule Networks for 3D Point Clouds
	   	作者 | Yongheng Zhao, Tolga Birdal, Jan Eric Lenssen, Emanuele Menegatti, Leonidas Guibas,\\ Federico Tombari\\
	   	单位 | 帕多瓦大学;斯坦福大学;慕尼黑工业大学;多特蒙德工业大学;谷歌\\
	   	论文 | https://arxiv.org/abs/1912.12098\\
	   	代码 | https://github.com/tolgabirdal/qecnetworks\\
	   	主页 | https://tolgabirdal.github.io/qecnetworks/\\
	   	备注 | ECCV 2020 Oral \\
	   	
	   	发明了新的3D胶囊网络,用于3D识别与方向估计。
	   	
	   	\item 
	   	[3].SoftPoolNet: Shape Descriptor for Point Cloud Completion and Classification\\
	   	作者 | Yida Wang, David Joseph Tan, Nassir Navab, Federico Tombari\\
	   	单位 | Technische Universit¨at M¨unchen;谷歌\\
	   	论文 | https://arxiv.org/abs/2008.07358\\
	   	备注 | ECCV 2020 Oral \\
	   	
	   	发明了新的用于点云的形状描述,用于点云补全和分类\\
	   	
	   	\item 
	   	[4].Combining Implicit Function Learning and Parametric Models for 3D Human Reconstruction\\
	   	作者 | Bharat Lal Bhatnagar, Cristian Sminchisescu, Christian Theobalt, Gerard Pons-Moll\\
	   	单位 | 萨尔大学;谷歌\\
	   	论文 | https://arxiv.org/abs/2007.11432\\
	   	代码 | https://github.com/bharat-b7/IPNet\\
	   	主页 | http://virtualhumans.mpi-inf.mpg.de/ipnet/\\
	   	备注 | ECCV 2020 Oral \\
	   	
	   	结合隐式函数与参数模型的3D人体重建。\\
	   	
	   	\item 
	   	[5].CoReNet: Coherent 3D scene reconstruction from a single RGB image\\
	   	作者 | Stefan Popov, Pablo Bauszat, Vittorio Ferrari\\
	   	单位 | 谷歌\\
	   	论文 | https://arxiv.org/abs/2004.12989\\
	   	备注 | ECCV 2020 Oral \\
	   	从单幅图片进行连贯的3D场景重建。\\
	   	
	   	\item 
	   	[6].Adversarial Generative Grammars for Human Activity Prediction\\
	   	作者 | AJ Piergiovanni, Anelia Angelova, Alexander Toshev, Michael S. Ryoo\\
	   	单位 | 谷歌;石溪大学\\
	   	论文 | https://arxiv.org/abs/2008.04888\\
	   	代码 | 即将\\
	   	备注 | ECCV 2020 Oral \\
	   	对抗生成语法用于人类活动预测。\\
	   	
	   	\item 
	   	[7].Self6D: Self-Supervised Monocular 6D Object Pose Estimation\\
	   	作者 | Gu Wang, Fabian Manhardt, Jianzhun Shao, Xiangyang Ji, Nassir Navab, Federico\\ Tombari\\
	   	单位 | 清华大学;慕尼黑工业大学;谷歌\\
	   	论文 | https://arxiv.org/abs/2004.06468\\
	   	代码 | https://github.com/THU-DA-6D-Pose-Group/Self6D-Diff-Renderer\\
	   	备注 | ECCV 2020 Oral \\
	   	自监督学习+单目6D位姿估计。\\
	   	
	   	\item 
	   	[8].What Matters in Unsupervised Optical Flow?\\
	   	作者 | Rico Jonschkowski, Austin Stone, Jonathan T. Barron, Ariel Gordon, Kurt Konolige,\\ Anelia Angelova\\
	   	单位 | 谷歌\\
	   	论文 | https://arxiv.org/abs/2006.04902\\
	   	代码 | https://github.com/google-research/google-research/tree/master/uflow\\
	   	备注 | ECCV 2020 Oral \\
	   	非监督光流估计研究。\\
	   	
	   	\item 
	   	[9].Appearance Consensus Driven Self-Supervised Human Mesh Recovery\\
	   	作者 | Jogendra Nath Kundu, Mugalodi Rakesh, Varun Jampani, Rahul Mysore Venkatesh, R.\\ Venkatesh Babu\\
	   	单位 | Indian Institute of Science, Bangalore;谷歌\\
	   	论文 | https://arxiv.org/abs/2008.01341\\
	   	代码 | 即将\\
	   	主页 | https://sites.google.com/view/ss-human-mesh\\
	   	备注 | ECCV 2020 Oral \\
	   	表观共识驱动的自监督人体网格修复。\\
	   	\item 
	   	[10].Fashionpedia: Ontology, Segmentation, and an Attribute Localization Dataset\\
	   	作者 | Menglin Jia, Mengyun Shi, Mikhail Sirotenko, Yin Cui, Claire Cardie, Bharath\\ Hariharan, Hartwig Adam, Serge Belongie\\
	   	单位 | 康奈尔大学;康奈尔科技校区;谷歌等\\
	   	论文 | https://arxiv.org/abs/2004.12276\\
	   	主页 | https://fashionpedia.github.io/home/index.html\\
	   	备注 | ECCV 2020 Oral \\
	   	定义了一个实例分割与细粒度属性定位的视觉新任务,并提出一个时尚领域的数据集。\\
	   \end{itemize}
	 

	
	
\end{document}