%导言区

\documentclass{ctexart}

\usepackage[a4paper,left=10mm,right=10mm,top=15mm,bottom=15mm]{geometry} 
\usepackage{listings}
\setlength{\parindent}{0pt}% 去掉首行缩进。
%\usepackage{titlesec}

% usepackage{ctex}
\title{windows链接服务器指南}
\date{\today}
\author{cuihu}

% 有一个宏包,可以自动产生pdf目录。

% 有一个宏包,可以自动产生superref 宏框的引用的超链接。




%正文区
\begin{document}
	\maketitle
	\section{命令行窗口进行链接}
	首先按确保win10安装了openshell\\
	打开命令行窗口。\\
	命令:ssh -p端口号 用户名@ip地址。\\
	例如:ssh -p22 cuihu@210.40.16.60\\
	回车,输入密码:cuihu614\\
	然后就链接如了ubuntu系统了。\\
	\section{安装anoconda}
	下载 conda 安装包,然后 bash .... 安装就可以了。\\
	安装的时候,是否添加到环境变量中,注意选择yes,因为默认是no\\
	手动添加方法:vim ~/.bashrc\\
	添加:\verb| export PATH='/home/用户名/anaconda3/bin:$PATH'|\\
	完成后: esc推出,按   :x (并推出。)\\
		当然,安装后记得source ~/.bashrc 激活环境变量。
	\section{安装虚拟环境}
	查看虚拟环境: conda info --envs \\
	新建虚拟环境: conda create --name xxx python=xxxx\\
	删除虚拟环境: conda remove -name xxx --all\\
	激活虚拟环境: conda activate xxx\\
	或者: . activate xxx\\
	推出虚拟环境: conda deactivate xxxx\\
	或者 . deactivate xxx\\
	
	\section{cuda,cudnn}
	nvcc -V\\
	cat /usr/local/cuda/version.txt\\
	 cat /usr/local/cuda/include/cudnn.h | grep CUDNN\_MAJOR -A 2 \\
	在服务器上安装cuda:\\
	下载cuda:\verb|cuda的下载地址: https://developer.nvidia.com/cuda-toolkit-archive|\\
	安装到指定的个人目录,安装驱动的时候选择no即可。\\
	参考文章:
	\verb|https://blog.csdn.net/sinat_20280061/article/details/80421532|\\
	\verb|https://blog.csdn.net/hizengbiao/article/details/88625044|\\
	下载cudnn:\verb|cuDNN的下载: https://developer.nvidia.com/rdp/cudnn-archive|\\
	
	\begin{lstlisting}
		sh cuda_8.0.61_375.26_linux.run #开始安装
		#..一堆协议说明...
		#直接按q退出协议说明.
		zerozone@zerozone: accept/decline/quit: accept  #接受协议
		
		Install NVIDIA Accelerated Graphics Driver for Linux-x86_64 375.26? 
		y)es/(n)o/(q)uit: n  #是否显卡驱动包,由于已经安装显卡驱动,选择n
		
		Install the CUDA 8.0 Toolkit?
		(y)es/(n)o/(q)uit: y #是否安装工具包,选择y
		
		Enter Toolkit Location
		[ default is /usr/local/cuda-8.0 ]: #工具包安装地址,
		这里我输入自己的安装地址:/home/zb/cuda/cuda-8.0/
		
		Do you want to install a symbolic link at /usr/local/cuda?
		(y)es/(n)o/(q)uit: n #添加链接**这里选择n
		
		Install the CUDA 8.0 Samples?
		(y)es/(n)o/(q)uit: y #安装样例,选择y
		
		Enter CUDA Samples Location
		[ default is /root ]:  #样例安装地址,
		这里我输入自己的安装地址:/home/zb/cuda/cuda-8.0/samples/
		
		环境变量:
		sudo vim ~/.bashrc
		添加:
		#cuda path --- install CUDA 10.0 ---
		export PATH="/home/sdb/614/cuihu/cuda/cuda_10.0/bin/:$PATH"
		export PATH=/home/sdb/614/cuihu/cuda/cuda_10.0/bin${PATH:+:${PATH}}
		export LD_LIBRARY_PATH=/home/sdb/614/cuihu/cuda/cuda_10.0/lib64${LD_LIBRARY_PATH:+:${LD_LIBRARY_PATH}}
		
		export CUDA_HOME=/usr/local/cuda
	\end{lstlisting}
	
	
	
	\section{cudnn}
	\begin{lstlisting}
		sudo cp cuda/include/cudnn.h   /home/sdb/614/cuihu/cuda/cuda_10.0/include
		sudo cp cuda/lib64/libcudnn*   /home/sdb/614/cuihu/cuda/cuda_10.0/lib64
		sudo chmod a+r /home/sdb/614/cuihu/cuda/cuda_10.0/include/cudnn.h
		sudo chmod a+r /home/sdb/614/cuihu/cuda/cuda_10.0/lib64/libcudnn*
	\end{lstlisting}
	
	\section{安装pytorch}
	\begin{lstlisting}
		-i https://mirrors.ustc.edu.cn/pypi/web/simple\\
		-i http://pypi.douban.com/simple/\\
		conda install pytorch==1.5.1 torchvision==0.6.1 cudatoolkit=10.1
		pip install torch==1.5.1+cu92 torchvision==0.6.1+cu92 -i 镜像
		pip install torch==1.5.1+cu91 torchvision==0.6.1+cu91 -i http://pypi.douban.com/simple/
		
		pip install torch==1.6.0+cu101 torchvision==0.7.0+cu101 -i https://pypi.tuna.tsinghua.edu.cn/simple
		10.1版本:\\
		在安装目录下\\
		
		如果不行的话,直接使用pytorch官网的的链接,服务器上的速度也很快。
		
		
		
		sh cuda_10.1.243_418.87.00_linux.run\\
	\end{lstlisting}
	
	
	\section{代码使用gpu指定}
	\begin{verbatim}
	1、每个GPU使用者都需进群,并修改自己备注(与服务器里的账户明一致)
	2、跑任何一个大(显存消耗)的程序都需要指定GPU运行,并指定空闲的GPU,\\
	以不影响他人的程序运行(通过nvidia-smi指令查看)。\\
	GPU指定方法:python程序需在程序里主代码添加
	(1、import os 2、os.environ['CUDA_VISIBLE_DEVICES'])='gpu_number');
	\\ Matlab程序需在主代码里添加(1、g=gpuDevice(gpu_number); reset(g);)\\
	
	
	3、当所有GPU均被使用时,可以与使用多个GPU的用户私下协商资源分配。\\
	
	4、正常情况下不要一个人占用了所有的GPU,\\
	除非在赶论文实验的(若他人也急需跑程序,应适当关掉一些进程)。\\
	注:此使用手册是为了使这有限的资源最充分的利用,谢谢大家的配合。\\
	祝:大家所有程序运行顺畅,实验得到理想结果
	\end{verbatim}

	\begin{verbatim}
		采取CUDA环境变量CUDA_VISIBLE_DEVICES来限定程序运行的GPU设备解决问题
		
		// CUDA_VISIBLE_DEVICES设置说明,设置device对程序可见
		CUDA_VISIBLE_DEVICES=1       // 仅使用device1 (即卡一)
		CUDA_VISIBLE_DEVICES=0,1     // 仅使用device 0和 device1
		CUDA_VISIBLE_DEVICES="0,1"	 // 同上, 仅使用device 0和 device1
		CUDA_VISIBLE_DEVICES=0,2,3   // 仅使用device 0, device2和device3
		CUDA_VISIBLE_DEVICES=2,0,3   // 仅使用device0, device2和device3
		
		#那么最后两条的区别是什么呢?
		CUDA_VISIBLE_DEVICES后面的参数依次是设置gpu[0],gpu[1], gpu[2]...等的device编号。
		所以区别在于: 0,2,3意思是gpu[0]指向device0, gpu[1], 指向devcie2, gpu[2]指向device3;
		而2,0,3意思是gpu[0]指向device2, gpu[1], 指向devcie0, gpu[2]指向device3;
		
		再举例说明,如果当前主机有5张显卡,默认情况下5个device对程序都可以见,默认排序device0 - 4。
		如果现在我们只希望使用第一张和第三张显卡,并且程序代码里看到的分别对应0,1。
		那么设置应该如下:
		CUDA_VISIBLE_DEVICES=0,2
	\end{verbatim}
	
	
	\section{}
	
\end{document}
