%导言区

\documentclass{ctexart}


% usepackage{ctex}
\usepackage{amsmath}

%正文区
\begin{document}
	\[
	\begin{matrix}
	0&1\\
	1&0
	\end{matrix}\qquad
	\begin{pmatrix}
	0&1\\
	1&0
	\end{pmatrix}\qquad
	\begin{bmatrix}
	0&1\\
	1&0
	\end{bmatrix}\qquad
	\begin{Bmatrix}
	0&1\\
	1&0
	\end{Bmatrix}\qquad
	\begin{vmatrix}
	0&1\\
	1&0
	\end{vmatrix}\qquad
	\begin{Vmatrix}
	0&1\\
	1&0
	\end{Vmatrix}\qquad
	\]
	
	%使用上下标。
	\[
	A = \begin{pmatrix}
	a_{11}^2 & a_{12}^2 & a_{13}^2\\
	0 & a_{22} & a_{23}\\
	0 & 0 & a_{33}
	\end{pmatrix}
	\]
	
	% 常用的省略号: \dots \vdots $\ddots
	\[
	a = \begin{bmatrix}
	a_{11} & \dots & a_{1n}\\
	& \ddots & \vdots \\
	0 & & a_{nn}
	\end{bmatrix}_{n \times n}	% times 命令排版乘号。
	\]
	
	%分块矩阵(矩阵的嵌套)
	\[
	\begin{pmatrix}
	\begin{matrix}
	1 & 0\\0& 1
	\end{matrix}%嵌套的矩阵,
	& \text{\Large 0} \\ %大写的0
	\text{\Large 0} &
	\begin{matrix}
	1&0\\0&1
	\end{matrix}% 嵌套的第二个矩阵。
	\end{pmatrix}
	\]
	
	% 三角矩阵。
	\[\begin{pmatrix}
	a_{11} & a_{12} & \cdots & a_{1n} \\
	& a_{22} & \cdots & a_{2n} \\
	&		 & \ddots & \vdots \\
	\multicolumn{2}{c}{\raisebox{1.3ex}[0pt]{\Huge 0}}
	&  		 &  a_{nn}
	\end{pmatrix}
	\]
	
	
	%跨列的省略号:\hdotsfor{列数}
	\[
	\begin{pmatrix}
	1 & \frac{1}{12} & \dots & \frac{1}{1n} \\
	\hdotsfor{4} \\ % 跨4列。
	m & \frac{1}{m2} & \dots & \frac{1}{mn}
	\end{pmatrix}
	\]
	
	% small matrix 行内小矩阵。
	复数 $z=(x,y)$ 也可用矩阵
	\begin{math}
		\left(
		\begin{smallmatrix}
		x & -y \\
		y & x
		\end{smallmatrix}
		\right) 
	\end{math}来表示。
	
	% array 环境。类似于表格环境tabular
	\[
	\begin{array}{r|r}
	\frac12 & 0 \\
	\hline
	0 & -\frac a{bc}
	\end{array}
	\]
	
	%用array黄静排版复杂的格式。
%	\[
%	\begin{array}{l}
%		% \left. 仅表示于\right\}配对,什么都不输出。
%		\left.\rule{0mm}{7mm}\right\}p\\
%		\\
%		\left.\rule{0mm}}{7mm}\right\}q	
%	\begin{array}
%	\\[-5pt]
%	% 第二行第一列。
%	\begin{array}{cc}
%	\underbrace{\rule{17mm}{0mm}}_m &
%	\underbrace{\rule{17mm}{0mm}}_m
%	\end{array}
%	& % 第二行第2列。
%	\end{array}
	
	
\end{document}