%导言
\documentclass{article}
\usepackage{ctex}

%正文
\begin{document}
	
故事:在现实认知观的基础上,对其描写成非常态性现象。是文学体裁的一种,侧重于事件发展过程的描述。强调情节的生动性和连贯性,较适于口头讲述。已经发生事。或者想象故事。

a\quad b % 一个1em ()当前字体中M的宽度。

a\qquad b 两个1em的宽度。

a\,b 

a\thinspace b
% 产生1/6个em。

a\enspace b
% 0.5em

a\ b
%空格。

a~b
产生一个硬空格。

a\kern 1pc b %1pc=12pt = 4.218mm

a\kern -1em b
指定间距大小。

a\hskip 1em b

a\hspace{35pt}b

%占位宽度
a\hphantom{xyz}b

%弹性长度空白
a\hfill b 产生弹性长度空白。用于充满整个空间。


%引号问题。
`hello' 单引号

双引号   ``hello''

\section{连字符}
- -- ---

\section{重音符}
\'o \`o \^o \~o \=o \.o

故事一般都和原始人类的生产生活有密切关系,他们迫切地希望认识自然,于是便以自身为依据,想象天地万物都像人一样,有着生命和意志。			
故事:在现实认知观的基础上,对其描写成非常态性现象。是文学体裁的一种,侧重于事件发展过程的描述。强调情节的生动性和连贯性,较适于口头讲述。已经发生事。或者想象故事。\par 故事一般都和原始人类的生产生活有密切关系,他们迫切地希望认识自然,于是便以自身为依据,想象天地万物都像人一样,有着生命和意志。



\end{document}