%导言区

\documentclass{ctexart}


% usepackage{ctex}
\usepackage{amsmath}
\usepackage{amssymb}
%正文区
\begin{document}
	% gather 和gather* 环境,代编号,每行公式都有。可以设置label
	\begin{gather}
		a+b=b+a\\
		ab = ba
	\end{gather}


	% 不代编号。
	\begin{gather*}
	a+b=b+a\\
	ab = ba
	\end{gather*}
	
	%也可以在 gather环境中使用 \notag 命令取消某一行的编号。
	\begin{gather}
	a+b=b+a\\
	ab = ba\\
	c = d\\
	ad =cd \notag
	\end{gather}
	
	%align 和align* 环境 (用&对齐)
	\begin{align}
		x &= t + \cos t + 1 \\ %等号左端对其。
		y &= 2\sin t
	\end{align}
	
	% 不代编号。
	\begin{align*}
		x &= t & x &= \cos t & x &= t\\
		y &= 2t & y&= \sin(t+1) & y &= \sin t
	\end{align*}
	
	%split环境,(对齐采用align环境方式,编号在中间。)
	\begin{equation} %因为是在equation环境下,所以就是一个公式。
	% 这也是equation和gather环境的不同之处。
		\begin{split}
		\cos 2x &= \cos^2 x - \sin^2 x \\
		&= 2\cos^2 x - 1
		\end{split}
	\end{equation}
	
	% cases环境
	%每行公式中使用& 分隔为两个部分。
	% 通常表示值和后面的条件。
	
	\begin{equation}
		D(X)=\begin{cases}
		1, & \text{如果} x \in \mathbb{Q};\\
		0, & \text{如果} x \in
		\mathbb{R}\setminus\mathbb{Q}.
		\end{cases}
	\end{equation}
%	数学公式中必须使用\text命令来切换中文,否则无法进行排版。
	
\end{document}
