%导言区
\documentclass{article}%如果不用宏包的话,
%\documentclass{ctexart}也可用来处理中文。
 %book,report,letter.ctexbook,ctexreport,
 %ctexletter.
\usepackage{ctex}%导入中文处理宏包。

\title{\heiti 勾股定理杂谈}
\author{\kaishu 张三}
\date{\today}
\newcommand\degree{^\circ}
%命令一般都是在导言区来定义的。实现命令和正文
%的分离。



%正文区(文稿区)

\begin{document}
	\maketitle


	
	Hello World.
	可以用符号语言表述为:设直角三角形$ABC$
	,其中$\angle C=90\degree$,则有:
	\begin{equation}	
		AB^2 = BC^2 + AC^2
	\end{equation}	%用equation方式定义
	%的公式,是有序号的。
	%here is my big formula
	Let $f(x)$ be defined by the formula 
	$f(x)=3x^2+x-1$
	$$f(x)=3x^2+x-1$$  %数学模式。没有公式号。
	which is a polynomial of degree 2.
\end{document}

%texdoc ctex在命令行窗口中打开,ctex手册。




